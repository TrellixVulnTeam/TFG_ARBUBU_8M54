\capitulo{1}{Introducción}

\section{Descripción del contenido del trabajo}
Ante la creciente demanda de las personas por tener identificado cada cosa que le rodea, surgió la idea de Arbubu.

Arbubu trata de poner en conocimiento de las personas los diferentes árboles singulares que se encuentran en las zonas universitarias, en un futuro podría expandirse a toda la ciudad de Burgos o a otras ciudades o incluso a otro tipo de ámbitos, pero de ello hablaremos más adelante en Conclusiones Lineas de trabajo futuras(añadir hiperenlace, cuando se cree).

Con todo esto tratamos de que los alumnos universitarios y las personas de todos los ámbitos y edades tengan a un solo click la información de los árboles singulares que les rodean.

Para facilitar la búsqueda de dichos árboles hemos incorporado un mapa interactivo que nos muestra en tiempo real donde están situados cada uno de ellos, con una ventana de información de la especie y del propio individuo.

Además para los más metidos en el tema de la naturaleza y los árboles se incorpora unos filtros de búsqueda, para que busquen un árbol en particular por el que están interesados.


\section{Estructura de la memoria}
La memoria se ha estructurado siguiendo los siguientes apartados:
\begin{itemize}
	\item Introducción: descripción del contenido del trabajo.
	\item Objetivos del Proyecto: explicación de los objetivos generales y técnicos del proyecto.
	\item Conceptos teóricos: explicación de los principales conceptos teóricos.
	\item Técnicas y herramientas: descripción breve y concisa de las técnicas y herramientas utilizadas en el proyecto.
	\item Aspectos relevantes del desarrollo del proyecto: explicación y desarrollo de los aspectos más relevantes del proyecto.
	\item Trabajos relacionados: descripción de trabajos que tengan cierto parecido al nuestro.
	\item Conclusiones lineas de trabajo futuras:descripción de las posibles líneas de trabajo futuras y conclusiones del proyecto.
\end{itemize}

\section{Estructura de los anexos}
\begin{itemize}
	\item Plan de proyecto: planificación temporal y estudio de la viabilidad económica y legal del proyecto.
	\item Requisitos: especificación de los requisitos que se establecen al principio del proyecto.
	\item Diseño: muestra la información relacionada con el diseño de a interfaz además del diseño de clases.
	\item Manual del programador: recoge la instalación de herramientas, la compilación, ejecución del proyecto y pruebas.
	\item Manual del usuario: guía de usuario con instrucciones que puedan facilitar el correcto manejo de la aplicación.
\end{itemize}

\section{Contenido del Cd}
\begin{itemize}
	\item Memoria: contenido de la memoria en formato pdf.
	\item Anexos: contenido de los anexos en formato pdf.
	\item Vídeo explicativo: vídeo explicando el funcionamiento básico de la aplicación web.
	\item Código: versión del código más reciente de la aplicación web.  	
\end{itemize}