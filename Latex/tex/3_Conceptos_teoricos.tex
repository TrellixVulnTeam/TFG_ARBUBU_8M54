\capitulo{3}{Conceptos teóricos}

En esta sección vamos a hablar de la forma en el cual vamos a clasificar nuestros modelos de datos, así como los conceptos teóricos relacionados con el proyecto.

\section{Estructura de datos}

Todos los seres vivos podemos clasificarnos según unas categorías taxonómicas \cite{CategoriaTaxonomica}, pero en este caso nos vamos a centrar en lo va el proyecto, que no es otro que los árboles.

Hemos elegido esta estructura de los datos porque con ellos somos capaces de englobar todos los aspectos que más se adecuan a nuestro proyecto.

No hemos decidido introducir más modelos de entidades ya que los comunes son los especificados y todo lo que fuera añadir alguna entidad más no nos permitiría desarrollar el proyecto de la mejor manera posible.

En nuestro proyecto vamos a clasificar los árboles de esta manera:
\begin{itemize}
	\item \textbf{Familia}: Es la agrupación de árboles que se encuentran en un orden, por características comunes entre ellos.
	\item \textbf{Género}: De las familias provienen los géneros, conjuntos de especies relacionadas entre sí por características comunes.
	\item \textbf{Especie}: Es un grupo de individuos con las mismas características.
	\item \textbf{Individuo}: Son cada uno de los árboles, con unas características particulares.
\end{itemize}

\section{Leaflet}

\textbf{Leaflet} \cite{leaflet} es una biblioteca JavaScript de código abierto utilizada para plasmar en nuestra página web mapas interactivos, los cuales son compatibles con dispositivos móviles.

Se caracteriza por la sencillez, simplicidad, rendimiento y usabilidad, lo que hace que sea una herramienta perfecta para nuestro proyecto.

También es posible añadirle infinidad de plugins que lo hace todavía más completo.

Para verlo con más claridad incluyo una imagen \ref{fig:conceptosTeoricos/leaflet} en la cual podemos observar un mapa con los países por los que ha pasado o vivido el titular del blog Andy Maloney.

\insertarimagen{conceptosTeoricos/leaflet}{Ejemplo mapa interactivo}{EjemploLeaflet}

