\capitulo{3}{Conceptos teóricos}

En esta sección vamos a hablar de la forma en el cual vamos a clasificar nuestros modelos de datos, así como los conceptos teóricos relacionados con el proyecto.

\section{Estructura de datos}

Todos los seres vivos podemos clasificarnos según unas categorías taxonómicas \cite{CategoriaTaxonomica}, pero en este caso nos vamos a centrar en lo va el proyecto, que no es otro que los árboles.

Hemos elegido esta estructura de los datos porque con ellos somos capaces de englobar todos los aspectos que más se adecuan a nuestro proyecto.

No hemos decidido introducir más modelos de entidades ya que los comunes son los especificados y todo lo que fuera añadir alguna entidad más no nos permitiría desarrollar el proyecto de la mejor manera posible.

En nuestro proyecto vamos a clasificar los árboles de esta manera:
\begin{itemize}
	\item \textbf{Familia}: Es la agrupación de árboles que se encuentran en un orden, por características comunes entre ellos.
	\item \textbf{Género}: De las familias provienen los géneros, conjuntos de especies relacionadas entre sí por características comunes.
	\item \textbf{Especie}: Es un grupo de individuos con las mismas características.
	\item \textbf{Individuo}: Son cada uno de los árboles, con unas características particulares.
\end{itemize}
\imagen{conceptosTeoricos/modelos}{Diagrama de Clases}
\section{Leaflet}

\textbf{Leaflet} \cite{leaflet} es una biblioteca JavaScript de código abierto utilizada para plasmar en nuestra página web mapas interactivos, los cuales son compatibles con dispositivos móviles.

Se caracteriza por la sencillez, simplicidad, rendimiento y usabilidad, lo que hace que sea una herramienta perfecta para nuestro proyecto.

También es posible añadirle infinidad de plugins que lo hace todavía más completo.

Para verlo con más claridad incluyo una imagen \ref{fig:conceptosTeoricos/leaflet} en la cual podemos observar un mapa con los países por los que ha pasado o vivido el titular del blog Andy Maloney.

\insertarimagen{conceptosTeoricos/leaflet}{Ejemplo mapa interactivo}{EjemploLeaflet}
\newpage

\section{Árboles Singulares}

No podemos pasar por alto el por qué vamos a hablar de estos árboles en nuestro proyecto y no centrarnos en otros distintos, diremos que todos los árboles que hemos elegido se diferencian del resto por un cúmulo de cosas que detallaremos a continuación:

\begin{itemize}
	\item \textbf{Antigüedad}: Diferenciaremos estos árboles del resto porque fueron plantados hace mucho tiempo y llevan conviviendo con nosotros varias generaciones.  
	\item \textbf{Elevado Diámetro}: Esta cualidad suele estar ligada a la antigüedad del árbol aunque no tiene por que ser siempre así, diremos que un árbol es singular por elevado diámetro cuando supera unas cantidades considerables.
	\item \textbf{Plantado por un personaje histórico}: Diferenciaremos estos árboles del resto porque fueron plantados por personajes que han sido importantes a lo largo de la historia de nuestra comunidad.
	\item \textbf{Vistosidad}: Estos árboles van a ser diferenciados del resto por la majestuosidad y preciosidad de su fachada, es decir, son todos aquellos árboles que ves y te quedas impresionado por lo bonitos que son. 
	\item \textbf{Madera codiciada}: Vamos a diferenciar estos árboles del resto porque históricamente se ha utilizado la madera de dichos árboles para diversas utilidades como la carpintería, el calentamiento de las casas \ldots
	\item \textbf{Frutos peculiares}: Todos estos árboles tienen una cualidad en común que es que sus frutos tienen peculiaridades tales como creencias antiguas que si los mezclabas con otros líquidos producían unas consecuencias o si los aprietas tienen un olor particular \ldots
	\item \textbf{Utilidad medicinal}: Estos árboles podemos diferenciarlos del resto ya que bien su semilla, hojas o su corteza han sido utilizados durante varias generaciones para la fabricación de medicamentos caseros que curaban o creía la gente que curaba desde una simple gripe a enfermedades más graves.
\end{itemize}
