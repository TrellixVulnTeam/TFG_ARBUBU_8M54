\capitulo{5}{Aspectos relevantes del desarrollo del proyecto}


En este apartado vamos a describir y comentar la formación que he necesitado y los distintos documentos, vídeos y demás materiales utilizados.

Además se explicarán los problemas surgidos para el correcto desarrollo del proyecto.


\section{Formación}

Como en todo proyecto en el cual nos embarcamos es necesario una formación extra que afiance los conocimientos adquiridos durante nuestra vida y más concretamente en nuestra estancia en el Grado de Ingeniería Informática.

Vamos a explicar paso a paso los distintos documentos de texto y gráficos consultados para el correcto aprendizaje de todas las herramientas utilizadas:

\begin{itemize}
	\item \textbf{Curso de Django de pildorasinformaticas} \cite{djangoPildoras}: Son una serie de vídeos en youtube donde un famoso youtuber y profesor explica desde el principio como crear un proyecto con Django e ir avanzando hasta cosas más complejas.
	\item \textbf{Tutorial Leaflet} \cite{tutorialLeaflet}: Son una serie de lecciones desde la configuración inicial hasta la realización de un mapa dinámico. 
	\item \textbf{Tutorial Python} \cite{tutorialPython}: Son una serie de lecciones desde como instalar Python a como instalar un módulo. 
	\item \textbf{Curso Django con Udemy} \cite{udemy}: Es un curso super básico de Django para aprender a programar páginas web.
	\item \textbf{Curso no convencional de LaTeX} \cite{cursoLatex}: Es un repositorio en GitHub que abarca numerosos contenidos de LaTeX. 
	\item \textbf{Curso CSS Avanzado} \cite{cursoCss}: Son una serie de vídeos en youtube donde un famoso youtuber y profesor explica desde el principio como crear un CSS e ir avanzando hasta cosas más complejas.
	\item \textbf{Curso HTML5} \cite{cursoHTML}: Son una serie de vídeos en youtube donde un famoso youtuber y profesor explica desde el principio como utilizar HTML5 e ir avanzando hasta cosas más complejas.
	\item \textbf{Como hacer un slideshow para un sitio web} \cite{slideshow}: Es un tutorial para aprender a hacer un slideshow en nuestra página web mediante HTML, CSS y Javascript.
	\item \textbf{Exportar Pdf} \cite{exportarPdf}: Se trata de una serie de pasos e indicaciones para obtener de un modelo los atributos de dicho modelo en Pdf.
	\item \textbf{Manual Django} \cite{manualDjango}: Es un manual de Django dividido en partes:
	\begin{itemize} 
		\item Parte 1: Comenzando por lo básico.
		\item Parte 2: Nivel avanzado.
		\item Parte 3: Baterías incluidas.
		\item Parte 4: Apéndices de referencia.
	\end{itemize}
	Es el mejor manual que he encontrado de Django, en el cual puedes consultar y eliminar cualquier duda.
\end{itemize}
\section{Problemas surgidos}

\subsection{Elección del TFG}

El primer problema surgido y principal fue la elección del Tfg, no porque el tema fuese malo, sino por el escaso tiempo del que disponía para su correcta puesta a punto, por diversos motivos hasta casi Abril no tuve el tema asignado y por eso me fue imposible presentarlo el curso anterior.

\subsection{Desconocimiento de diseño de páginas web}

Otro problema ocasionado era el desconocimiento de como empezar a realizar una página web, era un tema que siempre me ha gustado y me sigue gustando, pero que nunca había tocado con tanta cercanía.

He tenido que preguntar y pedir consejo a mis tutores, ver gran cantidad de documentos, vídeos, etc...
