\apendice{Plan de Proyecto Software}

\section{Introducción}

Todo proyecto considerado importante tiene que tener una buena planificación, en dicha planificación fijaremos los requisitos, el tiempo estimado y el dinero que creemos que va a costar realizar dicho proyecto.

El plan de proyecto software estará contenido por una planificación temporal y un estudio de viabilidad que nos dirá si el proyecto saldrá rentable o no.

\section{Planificación temporal}

Como decidí seguir una metodología en espiral \cite{ModeloenEspiral} el proyecto se divide en ciclos que a su vez se divide en 4 fases. 

\imagen{planProyecto/modeloEnEspiral}{Metodología en Espiral}

\subsection{Ciclo 1}

	\begin{enumerate}
		\item \textbf{Planificación}: Se decide en consenso con los tutores las herramientas a utilizar.
		\item \textbf{Análisis}: Al elegir una herramienta u otra se toman una serie de ventajas e inconvenientes y se decide optar por las que más me llaman la atención y las que considero como se explica en la memoria mas adecuadas para el proyecto.
		\item \textbf{Implementación}: Se instalan dichas herramientas y se empieza a trastear con ellas y se visualizan manuales para el correcto aprendizaje de su funcionamiento.
		\item \textbf{Evaluación}: Antes de pasar a otro ciclo se revisa las otras tres fases para ver si todo ha ido correctamente, se cree que las herramientas utilizadas tras manejarlas son las correctas y se cambia de ciclo.
	\end{enumerate}


\subsection{Ciclo 2}

\begin{enumerate}
	\item \textbf{Planificación}: Se tiene una reunión con los tutores para ver que es lo siguiente a realizar y se decide empezar por crear un proyecto y añadir documentación en Latex\cite{Latex}.
	\item \textbf{Análisis}: Se evalúa que lo que se va a realizar en este ciclo es lo correcto y se decide seguir con su desarrollo.
	\item \textbf{Implementación}: Se crea el proyecto en Django\cite{Django}, se crea los primeros modelos, se realiza el resumen y se añaden los objetivos del proyecto.
	\item \textbf{Evaluación}: Antes de pasar a otro ciclo se revisa las otras tres fases para ver si todo ha ido correctamente, se cree que el trabajo realizado es el correcto y se cambia de ciclo.
\end{enumerate}

\subsection{Ciclo 3}

\begin{enumerate}
	\item \textbf{Planificación}: Se tiene una reunión con los tutores para ver que es lo siguiente a realizar y se decide modificar los modelos.
	\item \textbf{Análisis}: Se evalúa que lo que se va a realizar en este ciclo es lo correcto y se decide seguir con su desarrollo.
	\item \textbf{Implementación}: Se modifica los modelos realizados anteriormente ante las exigencias del cliente.
	\item \textbf{Evaluación}: Antes de pasar a otro ciclo se revisa las otras tres fases para ver si todo ha ido correctamente, se cree que el trabajo realizado es el correcto y se cambia de ciclo.
\end{enumerate}

\subsection{Ciclo 4}

\begin{enumerate}
	\item \textbf{Planificación}: Se tiene una reunión con los tutores para ver que es lo siguiente a realizar y se decide crear unos métodos para ser utilizados más adelante.
	\item \textbf{Análisis}: Se evalúa que lo que se va a realizar en este ciclo es lo correcto y se decide seguir con su desarrollo.
	\item \textbf{Implementación}: Se crean unos métodos, pero cuando es enseñado al tutor se considera que no están bien realizados y se borran y se decide pasar a otro ciclo y dejar esto para más adelante.
	\item \textbf{Evaluación}: Antes de pasar a otro ciclo se revisa las otras tres fases y al considerar que la creación de los métodos no eran correctos se decide dejar para más adelante y se cambia de ciclo.
\end{enumerate}

\subsection{Ciclo 5}

\begin{enumerate}
	\item \textbf{Planificación}: Se tiene una reunión con los tutores para ver que es lo siguiente a realizar y se decide añadir el mapa y los primeros árboles.
	\item \textbf{Análisis}: Se evalúa que lo que se va a realizar en este ciclo es lo correcto y se decide seguir con su desarrollo.
	\item \textbf{Implementación}: Se añade el mapa y los primeros árboles.
	\item \textbf{Evaluación}: Antes de pasar a otro ciclo se revisa las otras tres fases para ver si todo ha ido correctamente, se cree que el trabajo realizado es el correcto y se cambia de ciclo.
\end{enumerate}

\subsection{Ciclo 6}

\begin{enumerate}
	\item \textbf{Planificación}: Se tiene una reunión con los tutores para ver que es lo siguiente a realizar y se decide avanzar con el diseño de la página y seguir avanzando con la documentación.
	\item \textbf{Análisis}: Se evalúa que lo que se va a realizar en este ciclo es lo correcto y se decide seguir con su desarrollo.
	\item \textbf{Implementación}: Se sigue avanzando en el diseño de la página y se crean los apartados de requisitos, técnicas metodológicas, conceptos teóricos, técnicas y herramientas y trabajos relacionados de la documentación.
	\item \textbf{Evaluación}: Antes de pasar a otro ciclo se revisa las otras tres fases para ver si todo ha ido correctamente, se cree que el trabajo realizado es el correcto y se cambia de ciclo.
\end{enumerate}

\subsection{Ciclo 7}

\begin{enumerate}
	\item \textbf{Planificación}: Se tiene una reunión con los tutores para ver que es lo siguiente a realizar y se decide crear la gestión de los usuarios.
	\item \textbf{Análisis}: Se evalúa que lo que se va a realizar en este ciclo es lo correcto y se decide seguir con su desarrollo.
	\item \textbf{Implementación}: Se crea la gestión de usuarios para poder iniciar sesión, registrarse y cerrar sesión.
	\item \textbf{Evaluación}: Antes de pasar a otro ciclo se revisa las otras tres fases para ver si todo ha ido correctamente, se cree que el trabajo realizado es el correcto y se cambia de ciclo.
\end{enumerate}

\subsection{Ciclo 8}

\begin{enumerate}
	\item \textbf{Planificación}: Se tiene una reunión con los tutores para ver que es lo siguiente a realizar y se decide ir introduciendo las fotos de los modelos en la base de datos, además de diseñar las pantallas de las especies, individuos, géneros y familias.
	\item \textbf{Análisis}: Se evalúa que lo que se va a realizar en este ciclo es lo correcto y se decide seguir con su desarrollo.
	\item \textbf{Implementación}: Se introduce las fotos de los modelos en la base de datos y se diseñan las pantallas de especies, individuos, géneros y familias.
	\item \textbf{Evaluación}: Antes de pasar a otro ciclo se revisa las otras tres fases para ver si todo ha ido correctamente, se cree que el trabajo realizado es el correcto y se cambia de ciclo.
\end{enumerate}

\subsection{Ciclo 9}

\begin{enumerate}
	\item \textbf{Planificación}: Se tiene una reunión con los tutores para ver que es lo siguiente a realizar y se decide ir introduciendo los árboles en los distintos mapas de la aplicación a través de ficheros javascript.
	\item \textbf{Análisis}: Se evalúa que lo que se va a realizar en este ciclo es lo correcto y se decide seguir con su desarrollo.
	\item \textbf{Implementación}: Se introducen los árboles en los distintos mapas mediante llamadas a ficheros javascript.
	\item \textbf{Evaluación}: Antes de pasar a otro ciclo se revisa las otras tres fases para ver si todo ha ido correctamente, se cree que el trabajo realizado es el correcto y se cambia de ciclo.
\end{enumerate}

\subsection{Ciclo 10}

\begin{enumerate}
	\item \textbf{Planificación}: Se tiene una reunión con los tutores para ver que es lo siguiente a realizar y se decide agregar una opción para añadir individuos mediante formularios y exportarlos en pdf.
	\item \textbf{Análisis}: Se evalúa que lo que se va a realizar en este ciclo es lo correcto y se decide seguir con su desarrollo.
	\item \textbf{Implementación}: Se introduce una opción para que un usuario registrado pueda añadir individuos y también para que pueda descargar un pdf con los árboles y sus características y localización.
	\item \textbf{Evaluación}: Antes de pasar a otro ciclo se revisa las otras tres fases para ver si todo ha ido correctamente, se cree que el trabajo realizado es el correcto y se cambia de ciclo.
\end{enumerate}

\subsection{Ciclo 11}

\begin{enumerate}
	\item \textbf{Planificación}: Se tiene una reunión con los tutores para ver que es lo siguiente a realizar y una vez terminada el diseño de las pantallas de aplicación se decide seguir con la documentación de los anexos.
	\item \textbf{Análisis}: Se evalúa que lo que se va a realizar en este ciclo es lo correcto y se decide seguir con su desarrollo.
	\item \textbf{Implementación}: Se continúa con la documentación de los anexos.
	\item \textbf{Evaluación}: Antes de pasar a otro ciclo se revisa las otras tres fases para ver si todo ha ido correctamente, se cree que el trabajo realizado es el correcto y se cambia de ciclo.
\end{enumerate}

\section{Estudio de viabilidad}

En este apartado se va a analizar la viabilidad económica y la viabilidad legal.


\subsection{Viabilidad económica}

Si se implantase el desarrollo en un entorno empresarial real los costes serían los siguientes:

\begin{itemize}
	\item \textbf{Costes de personal}: Teniendo en cuenta que el proyecto ha tenido una duración aproximada de 6 meses, considerando que lo lleva a cabo un desarrollador con un salario bruto de 1300\textup{\euro} mensuales contratado a tiempo parcial, con unas contingencias comunes\cite{seguridadSocial} de 28.3\%  y una retención del IRPF de 10.49\%, el salario neto será de 795.73\textup{\euro} y el coste total será de 7800\textup{\euro}.
	
	\item \textbf{Costes de material}: Hay que tener en cuenta el coste de hardware y coste de software, el coste del software es gratuito ya que como hemos mencionado en la memoria todas las herramientas utilizadas han sido de software libre y gratuito, el coste del hardware es la utilización de un portátil, contando que el coste del ordenador fue de 600\textup{\euro} y con una antigüedad de aproximadamente 4 años y el tiempo utilizado de 6 meses, el coste amortizado sería de 60\textup{\euro}.
	
	\item \textbf{Costes de material}: La suma del coste de personal y los costes de material es de 7860\textup{\euro}.

\end{itemize}

	Teniendo en cuenta los resultados obtenidos podemos concluir que el proyecto resultará rentable a partir de unos dos años de vida.
	
\subsection{Viabilidad legal}

	En este apartado se expondrá la viabilidad legal del proyecto, al utilizar Python que posee licencia PSFL\cite{licencia}, es de software libre y cumple los requisitos OSI y es compatible con licencia GPL.
	
	Además todo el software utilizado ha sido de software libre, así que la licencia que se adapta mejor al proyecto es la GNU\cite{gnu}, esta licencia permite el uso, distribución y modificación siempre y cuando no se modifique la dicha licencia y acredite al autor.
	
	