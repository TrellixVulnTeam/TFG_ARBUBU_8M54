\capitulo{2}{Objetivos del proyecto}

El principal objetivo del proyecto es realizar un diseño web, en el cual se puedan ver los árboles singulares de las zonas universitarias de Burgos, con sus principales características.

A través de un mapa podremos ver donde están ubicados los árboles. 
\section{Objetivos Generales}
\begin{itemize}
	\item Observar en un mapa los árboles singulares de las zonas universitarias de Burgos y ver sus características.
	\item Filtrar a través de la familia, género, especie y ubicación los distintos árboles.
	\item Loguearnos como usuario y ser capaces de introducir y descargar datos de los árboles.
	\item Realizar una primera toma de contacto con la búsqueda de árboles, que en una futura mejora no solo busquemos arboles, es decir, que seamos capaces de buscar monumentos, lugares importantes \ldots
	\item Poder compartir y dar me gusta a la página de facebook de UbuVerde e interactuar con ellos a través de twitter. 
	\item Estar en constante contacto con UbuVerde a través de 'Acerca de Nosotros' que es un enlace a la página de UbuVerde.
	
	\imagen{objetivos/portada}{Portada de la página web}
\end{itemize}
\newpage

\section{Objetivos Técnicos}
\begin{itemize}
	\item Crear la Base de Datos para guardar la información de las familias, géneros, especies e individuos con una estructura adecuada de tablas y campos.
	\item Ser capaz de introducir nuevos individuos en la base de datos Sqlite3.
	\item Ser capaz de descargar información de la base de datos Sqlite3.
	\item Plasmar en el mapa esos datos introducidos en la base de datos. 
	\item Desarrollar el diseño web con Django quedando plasmado con una interfaz gráfica amena y agradable para el usuario.
	\item Guardar en un repositorio de GitHub los cambios que hemos ido realizando.
	 
\end{itemize}