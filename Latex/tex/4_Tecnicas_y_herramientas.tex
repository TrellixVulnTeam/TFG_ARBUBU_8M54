\capitulo{4}{Técnicas y herramientas}
Esta sección vamos a hablar de las técnicas metodológicas  y de las herramientas de desarrollo seguidas durante el proyecto.

\section{Técnicas Metodológicas}
No hemos seguido una metodología pura, es decir, no me he basado simplemente en una sola metodología si no que he ido eligiendo aspectos de varias de ellas.

En primer lugar, tomamos aspectos de la metodología en cascada\cite{ModeloenCascada}, ya que hemos partido de unos requisitos iniciales, que posteriormente hemos ido adaptando según los cambios y necesidades del cliente.

Más adelante una vez que teníamos claros los requisitos iniciales, nos hemos centrado más en la metodología scrum\cite{MetodologiaScrum}. Decimos que utizamos esta metodologia porque a lo largo del desarrollo del proyecto hemos tenido reuniones semanales con los tutores, además de una reunión inicial con el encargado de Ubu Verde.

En todo momento hemos ido mezclando ambas técnicas ya que a las reuniones semanales con los tutores se fueron añadiendo a continuación el diseño e implementación de la aplicación.

Una vez que el diseño e implementación estaban realizados fuimos añadiendo las distintas pruebas para comprobar que no dejábamos cabos sueltos.

A medida que íbamos realizando las distintas partes del proyecto se iban subiendo al repositorio de Github.

\section{Herramientas de Desarrollo}



 Si se han estudiado diferentes alternativas de metodologías, herramientas, bibliotecas se puede hacer un resumen de los aspectos más destacados de cada alternativa, incluyendo comparativas entre las distintas opciones y una justificación de las elecciones realizadas. 
No se pretende que este apartado se convierta en un capítulo de un libro dedicado a cada una de las alternativas, sino comentar los aspectos más destacados de cada opción, con un repaso somero a los fundamentos esenciales y referencias bibliográficas para que el lector pueda ampliar su conocimiento sobre el tema.
