\capitulo{7}{Conclusiones y Líneas de trabajo futuras}

\section{Conclusiones} 

Tras haber acabado con el proyecto se puede decir que ha cumplido la gran mayoría de objetivos y requisitos que nos habían propuesto.

Como consecuencia del desarrollo del proyecto he aprendido a utilizar herramientas como Django, SQLite 3 y he perfeccionado mis conocimientos en Python, Css y HTML.

Otro tema muy importante para el desarrollo de la aplicación web es la utilización de servidores para alojarla en la nube.

Por otra parte, he aprendido gran cantidad de aspectos relacionados con los árboles y me ha parecido un tema muy ameno y con el que no me importaría seguir trabajando en un futuro.

Para finalizar puedo afirmar que estoy muy contento con el trabajo realizado, siendo una experiencia muy positiva académica y personalmente ya que he trabajado con unos tutores que me han ayudado en todo lo que han podido.

\section{Líneas de trabajo futuras} \label{trabajosFuturos}

Este proyecto puede ser tan grande como el diseñador y programador quiera, es decir, no tiene por qué limitarse a ser un proyecto basado únicamente en árboles, puede basarse en animales avistados, monumentos de gran patrimonio cultural, bares históricos \ldots 

Lógicamente únicamente con los modelos creados no podríamos abarcar todos estos temas anteriores, habría que crear nuevos modelos de datos y reutilizar los que si nos podrían valer de los ya realizados.

A medida que el proyecto fuese creciendo veríamos que la base de datos elegida (SQLite 3) se va quedando pequeña, ya que no admite una gran cantidad de datos y sería necesario cambiar a otra como Oracle o MySql, que no supondría ningún inconveniente, ya que es cambiar unas pequeñas líneas de código.  

Otra posible mejora es la internacionalización de la página web, es decir, que pueda cambiarse de un idioma a otro según la procedencia de la persona que está buscando la información.

 
